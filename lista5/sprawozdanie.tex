\documentclass{article}
\title{Analiza algorytmów. Lista 4}
\author{Piotr Berezowski, 236749}

\usepackage{polski}
\usepackage[utf8]{inputenc}
\usepackage{enumerate}
% \usepackage{subfig}
\usepackage{amsmath}
\usepackage{amsfonts}
\usepackage{cleveref}
\usepackage{cases}
\usepackage{mathtools}
\usepackage{float}
\usepackage{graphicx}
\usepackage{caption}
\usepackage{subcaption}
\usepackage[ruled,vlined,linesnumbered,longend]{algorithm2e}
\graphicspath{ {./src/} }

\newenvironment{pseudokod}[1][htb]{
	\renewcommand{\algorithmcfname}{}
	\begin{algorithm}[#1]%
	}{
\end{algorithm}
}

\begin{document}
	\maketitle
	\pagenumbering{gobble}
	\newpage
    \pagenumbering{arabic}
    

	\section{Zadanie 12}
	\subsection{Opis zadania}
    Zaimplementuj symulator algorytmu Mutual Exclusion Dijkstry. Dla ustalonego $n$ oznaczającego liczbę procesów w pierścieniu, zweryfikuj, 
    że startując z dowolnej konfiguracji początkowej algorytm przejdzie do legalnej konfiguracji. Jeśli z pewnej konfiguracji można przejść 
    do kilku możliwych konfiguracji w zależności od tego, który proces wykona krok jako pierwszy, każde wykonanie powinno zostać zweryfikowane. 
    Jaka jest największa liczba kroków do czasu osiągnięcia legalnej konfiguracji dla ustalonego $n$? Dla jakich wartości $n$ możesz uzyskać 
    odpowiedź w sensownym czasie? Za zadanie możesz otrzymać $3 \times N$ punktów, gdzie $N$ oznacza największą wartość $n$, dla której uda Ci się 
    zweryfikować algorytm.
    
    \subsection{Rozwiązanie}

    Implementacja zadania znajduje się w pliku \textit{zad1.py}. 

    Największa wartość $n$ dla jakiej algorytm udało się zweryfikować jest $n = 7$, gdzie suma wszystkich możliwych konfiguracji 
    jest równa $2097152$. Największa liczba kroków do legalnej konfiguracji dla takiego 
    $n$ jest równa $57$.

    Poniżej w tabeli przedstawiono wyniki dla kolejnych wartości $n$, dla których udało się zweryfikować algorytm.

    \begin{table}[H]
        \begin{center}
            \resizebox{\textwidth}{!}{%
            \begin{tabular}{c||c|c|c}
                \textbf{n} & \textbf{Ilość wszystkich konfiguracji} & \textbf{Ilość wszystkich możliwych przejść} & \textbf{Max liczba kroków} \\
                \hline

                1 & 2 & 0 & 0 \\
                2 & 9 & 6 & 1 \\
                3 & 64 & 108 & 4 \\
                4 & 625 & 1620 & 15 \\
                5 & 7776 & 27210 & 26 \\
                6 & 117649 & 521010 & 40 \\
                7 & 2097152 & 11272184 & 57 \\
                
            \end{tabular}
            }
        \end{center}
        \caption{Wyniki dla przypadków testowych z pliku \textit{gapd.txt}. (*) $cap / cap_{max}$ przedstawia stosunek użytej pojemności maszyny 
            w wyznaczonym rozwiązaniu do maksymalnej pojemności maszyny. W tabeli przedstawiono minimalny, średni i maksymalny stosunek.}
        \label{wyniki-d}
    \end{table}


    \section{Zadanie 13}
	\subsection{Opis zadania}
    Rozważmy graf $G= (V,E)$. Dwa wierzchołki $v,w \in V$ nazywamy niezależnymi, jeśli $\{v,w\} \notin E$. Podzbiór $S \subseteq V$ wierzchołków 
    nazywamy niezależnym, jeśli wszystkie jego elementy są parami niezależne. Wzorując się na algorytmie Maximal Matching podanym na wykładzie 
    zaprojektuj, zaimplementuj i przetestuj samostabilizujący algorytm znajdujący maksymalny zbiór niezależny (ang. Maximal Independent Set) 
    w nieskierowanym grafie spójnym. Podaj przekonywujące uzasadnienie poprawności algorytmu (formalny dowód - zadanie na ćwiczenia). 
    Algorytmy znajdowania maksymalnego zbioru niezależnego mają wiele zastosowań, możesz np. myśleć o problemie przydziału częstotliwości 
    w sieciach bezprzewodowych.
    
    \subsection{Rozwiązanie}

    Implementacja zadania znajduje się w pliku \textit{zad2.py}.

    Każdy proces odpowiada pojedynczemu wierzchołkowi w grafie. Każdy proces kontroluje jeden rejestr $r_p \in \{0, 1\}$ który przechowuje 
    informacje o tym, czy dany wierzchołek należy do zbioru niezależnego. $N(p)$ oznacza zbiór sąsiadów $p$. 
    W każdym kroku algorytmu, dla wierzchołka $p$ możemy znajdować się w jednej z następujących sytuacji:
    \begin{enumerate}
        \item Wszystkie wierzchołki $q \in N(p)$ spełniają $r_q = 0$.
        \item Przynajmniej jeden wierzchołek $q \in N(p)$ spełnia $r_q = 1$.
    \end{enumerate}

    W przypadku sytuacji 1, jeśli $r_p = 0$, to ustawiamy $r_p \gets 1$, w przeciwnym przypadku nie robimy nic.

    W przypadku sytuacji 2, jeśli $r_p = 1$, to ustawiamy $r_p \gets 0$, w przeciwnym przypadku nie robimy nic.
    
    Po ustabilizowaniu się algorytmu (wyznaczeniu maksymalnego zbioru niezależnego) żaden jego krok nie zmieni obecnej konfiguracji. 
    Wszyscy sąsiedzi wierzchołków oetykietowanych numerem $1$ będą miały ustawioną wartość rejestru na 0 (sytuacja 2), a wszyscy sąsiedzi 
    wierzchołków oetykietowanych numerem $0$ będą miały tylko jednego sąsiada z wartością rejestru równą 1. Widzimy, że w tym przypadku 
    wierzchołki, które zostały oetykietowane numerem $1$ tworzą pewien maksymalny zbiór niezależny.

    Poniżej przedstawiono pseudokod pętli wykonywanej przez każdy z procesów.

    \begin{pseudokod}[H]
        \caption{Każdy proces $p$ wykonuje pętle}
        \While{True} {
            \BlankLine
            \If{$r_p = 0 \land (\forall_{q \in N(p)}) (r_q = 0)$} {
                $r_p \gets 1$\;
            }
            \If{$r_p = 1 \land (\exists_{q \in N(p)}) (r_q = 1)$} {
                $r_p \gets 0$\;
            }
        }
    \end{pseudokod}

\end{document}