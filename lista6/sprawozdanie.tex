\documentclass{article}
\title{Analiza algorytmów. Lista 6}
\author{Piotr Berezowski, 236749}

\usepackage{polski}
\usepackage[utf8]{inputenc}
\usepackage{enumerate}
% \usepackage{subfig}
\usepackage{amsmath}
\usepackage{amsfonts}
\usepackage{cleveref}
\usepackage{cases}
\usepackage{mathtools}
\usepackage{float}
\usepackage{graphicx}
\usepackage{caption}
\usepackage{subcaption}
\usepackage[ruled,vlined,linesnumbered,longend]{algorithm2e}
\graphicspath{ {./src/} }

\newenvironment{pseudokod}[1][htb]{
	\renewcommand{\algorithmcfname}{}
	\begin{algorithm}[#1]%
	}{
\end{algorithm}
}

\begin{document}
	\maketitle
	\pagenumbering{gobble}
	\newpage
    \pagenumbering{arabic}
    

	\section{Zadanie 14}
	\subsection{Opis zadania}
    Układając odpowiednie równanie rekurencyjne i wykorzystując funkcje tworzące dla danego $n$ wyznacz liczbę wywołań linii $6$ w poniższym 
    algorytmie. Zweryfikuj odpowiedź eksperymentalnie.

    \begin{pseudokod}[H]
        \caption{$f(\text{int } n)$}

        $\text{int } s = 0$\;
        \If{$n == 0$} {
            \Return{1}
        }
        \Else{
            \For{$\text{int } i = 0; i < n; i++$}{
                $s += f(i)$\;
            }
            \Return{s}
        }

    \end{pseudokod}

    \subsection{Rozwiązanie}

    Niech $F_{n}$ oznacza liczbę wywołań lini $6$ dla $n$. Dla $n = 0$ linia $6$ nie wykona się ani razu, więc $F_{0} = 0$. 

    W momencie kiedy $n > 0$ linia $6$ wykonuje się rekurencyjnie dla kolejnych $i < n$:

    $$F_{n} = \sum_{i=0}^{n-1} (F_{i} + 1) = n + \sum_{i=0}^{n-1} F_{n}, \ \ \text{dla } n > 0$$

    Zapiszmy różnicę $F_{n+1}$ i $F_{n}$:

    $$F_{n+1} - F_{n} = n + 1 + \sum_{i=0}^{n} F_{i} - n - \sum_{i=0}^{n-1} F_{i}$$

    $$F_{n+1} = 1 + 2 F_{n}$$

    Wyznaczmy teraz funkcję tworzącą $F(z)$ ciągu $F_{n}$:

    \begin{equation*}
        \begin{split}
            F(z) = \sum_{n \geq 0} F_{n} z^{n} = F_{0} + \sum_{n \geq 1} F_{n} z^{n} = \sum_{n \geq 0} F_{n+1} z^{n+1} = 
                z \sum_{n \geq 0} (1 + 2 F_{n}) z^{n} = \\ 
                2 z \sum_{n \geq 0} F_{n} z^{n} + z \sum_{n \geq 0} z^{n} = 
                2 z F(z) + z \sum_{n \geq 0} z^{n} = 2 z F(z) + \frac{z}{1 - z}
        \end{split}
    \end{equation*}


    $$F(z)(1 - 2z) = \frac{z}{1 - z}$$

    $$F(z) = \frac{z}{(1 - z)(1 - 2z)} = \frac{1}{1 - 2z} - \frac{1}{1 - z} = \sum_{n \geq 0} 2^{n} z^{n} - \sum_{n \geq 0} z^{n} = 
        \sum_{n \geq 0} (2^{n} - 1) z^{n}$$

    Ostatecznie:

    $$[z^{n}] F(z) = F_{n} = 2^{n} - 1$$

    Wyniki dla kolejnych n:

    \begin{table}[H]
        \begin{center}
            % \resizebox{\textwidth}{!}{%
            \begin{tabular}{c||c|c}
                \textbf{n} & \textbf{\textbf{$F_n$}} & \textbf{Wynik eksperymentalny} \\
                \hline

                0 & 0 & 0 \\
                1 & 1 & 1 \\
                2 & 3 & 3 \\
                3 & 7 & 7 \\
                4 & 15 & 15 \\
                5 & 31 & 31 \\
                6 & 63 & 63 \\
                7 & 127 & 127 \\
                8 & 255 & 255 \\
                9 & 511 & 511 \\
                10 & 1023 & 1023 \\
                11 & 2047 & 2047 \\
                12 & 4095 & 4095 \\
                13 & 8191 & 8191 \\
                14 & 16383 & 16383 \\
                15 & 32767 & 32767 \\
                16 & 65535 & 65535 \\
                17 & 131071 & 131071 \\
                18 & 262143 & 262143 \\
                19 & 524287 & 524287 \\
                20 & 1048575 & 1048575 \\
                21 & 2097151 & 2097151 \\
                22 & 4194303 & 4194303 \\
                23 & 8388607 & 8388607 \\
                24 & 16777215 & 16777215 \\
                25 & 33554431 & 33554431 \\
                26 & 67108863 & 67108863 \\
                27 & 134217727 & 134217727 \\
                28 & 268435455 & 268435455 \\
                29 & 536870911 & 536870911 \\
                30 & 1073741823 & 1073741823 \\
                
            \end{tabular}
            % }
        \end{center}
    \end{table}

    \section{Zadanie 15}
    \subsection{Opis zadania}
    Algorytm otrzymuje na wejściu tablicę długości $n \geq 0$. Jeśli $n \geq 2$ dla każdego $k \in \{1,2,3,...,n\}$ algorytm z prawdopodobieństwem 
    $1/2$ wywołuje się rekurencyjnie na pewnej losowej „podtablicy” długości $k$. Wykorzystując funkcje tworzące wyznacz średnią liczbę 
    wywołań algorytmu dla danego $n$ i przedstaw swoje wyliczenia. Zweryfikuj odpowiedź eksperymentalnie.

    
    \subsection{Rozwiązanie}

    
    
\end{document}